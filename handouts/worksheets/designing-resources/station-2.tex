\documentclass[11pt]{article}
\usepackage{./cs-370}

\def\title{Station 1}

\begin{document}

\maketitle

% Name

\name

% Instructions

At this station, you will design questions that target the higher levels of Bloom's taxonomy while relying on the lower levels for support. On the following pages, you'll find 4 challenging problems from past CS 61A midterms and final exams. For each problem, complete the two tasks below. Don't feel compelled to go in any particular order, and please complete \textit{both} tasks for each question before moving on.

% Questions

\begin{qlist}

\q{Task 1}
Identify where the problem falls on Bloom's pyramid. Remember, the categories are \textit{Remember}, \textit{Understand}, \textit{Apply}, \textit{Analyze}, and \textit{Create}. Discuss this in depth with your teammates! Multiple answers may be correct.

\q{Task 2}
Come up with a set of questions that you could use to lead a student through the problem, using the Socratic method. Assume no skeleton code is provided. Remember to go high-level first, and then expand on the details. Also think about how you or your team would approach solving the problem.

\end{qlist}

\newpage

\begin{qlist}

\q{MULTIADDER}
Implement \texttt{multiadder}, which takes a positive integer \texttt{n} and returns an order $n$ numeric function that sums an argument sequence of length $n$.
\begin{verbatim}
def multiadder(n):
    """Return a function that takes n arguments, one at a time, and adds them.
    >>> f = multiadder(3)
    >>> f(5)(6)(7)                # 6 + 7
    18
    >>> multiadder(1)(5)          # 5
    5
    >>> multiadder(2)(5)(6)       # 5 + 6
    11
    >>> multiadder(4)(5)(6)(7)(8) # 5 + 6 + 7 + 8
    26
    """
    if _______________________________________________________________:
        return ________________________________________________________
    else:
        return ________________________________________________________

def multiadder(n):
    if n == 1:
        return lambda x: x
    else:
        return lambda a: lambda b: multiadder(n-1)(a+b)
\end{verbatim}

\newpage

\q{PILE}
A \textit{pile} for a tree $t$ with no repeated leaf labels is a dictionary in which the label for each leaf of $t$ is a key, and its value is the path from that leaf to the root. Each path from a node to the root is either an empty tuple, if the node is the root, or a two-element tuple containing the label of the node's parent and the rest of the path. Implement \texttt{pile}, which takes a tree constructed using a tree constructed with the \texttt{tree} data abstraction. It returns a \textit{pile} for that tree. You may use the \texttt{tree}, \texttt{label}, \texttt{branches}, and \texttt{is\_leaf} functional abstractions.
\begin{verbatim}
def pile(t):
    """Return a dict that contains every path from a leaf to the root of t.
    >>> pile(tree(5, [tree(3, [tree(1), tree(2)]), tree(6, [tree(7)])]))
    {1: (3, 5, ()), 2: (3, 5, ()), 7: (6, (5, ()))}
    """
    p = {}
    def gather(__________________________, __________________________):
        if is_leaf(u):
            ___________________________________________________________
        for b in branches(u):
            ___________________________________________________________
    ___________________________________________________________________
    return p

def pile(t):
    """Return a dict that contains every path from a leaf to the root of t.
    >>> pile(tree(5, [tree(3, [tree(1), tree(2)]), tree(6, [tree(7)])]))
    {1: (3, 5, ()), 2: (3, 5, ()), 7: (6, (5, ()))}
    """
    p = {}
    def gather(u, parent):
        if is_leaf(u):
            p[label(u)] = parent
        for b in branches(u):
            gather(b, (label(u), parent))
    gather(t, ())
    return p
\end{verbatim}

\newpage

\q{RE-SET}
Implement the \texttt{fset} function, which returns two functions that together represent a set. Both the \texttt{add} and \texttt{has} functions return whether a value is already in the set. The \texttt{add} function also adds its argument value to the set. You may assign to only one name in the assignment statement. You may not use any built-in container, such as a set or dictionary or list.
\begin{verbatim}
def fset():
    """Return two functions that together represent a set.
    >>> add, has = fset()
    >>> [add(1), add(3)]               # Neither 1 nor 3 were already added.
    [False, False]
    >>> [has(k) for k in range(5)]
    [False, True, False, True, False]
    >>> [add(3), add(2)]              # 3 was already in the set; 2 is added.
    [True, False]
    >>> [has(k) for k in range(5)]
    [False, True, True, True, False]
    """
    items = lambda x: _________________________________________________
    def add(y):
        _______________________________________________________________
        f = items
        _________ = ___________________________________________________
        return ________________________________________________________
    return add, _______________________________________________________

def fset():
    items = lambda x: False
    def add(y):
        nonlocal items
        f = items
        items = lambda x: x == y or f(x)
        return f(y)
    return add, lambda y: items(y)

\end{verbatim}

\end{qlist}

\end{document}
